\documentclass[10pt,letterpaper]{article}
\usepackage[utf8]{inputenc}
\usepackage[spanish]{babel}
\usepackage{amsmath}
\usepackage{amsfonts}
\usepackage{amssymb}
\usepackage{graphicx}
\usepackage{enumitem}
\usepackage{hyperref}
\usepackage[left=2cm,right=2cm,top=2cm,bottom=2cm]{geometry}
\author{Edward Villegas}
\title{Radiación de cuerpo negro numérica}
\begin{document}
El problema de la radiación de cuerpo negro esta descrito por la ley de Planck que es
\begin{equation}
B_\nu(\nu, T) = \frac{2h\nu^3}{c^2}\frac{1}{\exp\left(\frac{h\nu}{k_BT} -1\right)}. \label{eq:planck_v}
\end{equation}
\begin{enumerate}
\item Para 5 valores diferentes de temperatura, determine la frecuencia a la cual se presenta la máxima emisividad.

\textbf{Tip}: Optimización / Métodos basados en gradiente.

\item Suponga, que las frecuencias de máxima emisividad cumplen una relación lineal con respecto a la temperatura. Usando el método de mínimos cuadrados, determine los coeficientes del polinomio lineal que determina esta relación. Compare el resultado con la ley de desplazamiento de Wien.

\textbf{Tip}: Regresiones lineales / Solución iterativa de sistemas de ecuaciones lineales.

\item Obtenga la aproximación de Rayleigh Jeans para bajas frecuencias $\nu \rightarrow 0$.

\textbf{Tip}: Usando 3 valores de $\nu$, incluyendo $\nu=0$, anteriores a la frecuencia máxima a una temperatura fija, aplique interpolación de Newton (equivalente a la expansión de series de Taylor) o diferencia central en reemplazo de la derivada para la serie de Taylor.

\item Obtenga numéricamente la constante de Stefan-Boltzmann, $\sigma$. 

\textbf{Tip}: Para los 5 valores de temperatura usados, integrar ley de Planck sobre todas las frecuencias y angulo solido que determina una semiesfera. Esta consideración lleva al problema
$$ j^* = \frac{2 \pi h}{c^2} \int_0^\infty \frac{\nu^3}{\exp\left(\frac{h\nu}{kT} \right)-1} d\nu = \sigma T^4. $$
Tras un cambio de variable, es posible resolver el problema equivalente
$$ j^* = \left(\frac{2 \pi k^4}{c^2 h^3} \int_0^\infty \frac{u^3}{\exp(u)-1} du \right) T^4 = \sigma T^4. $$

\item Aproxime la constante de Stefan-Boltzmann partiendo de una interpolación de la la ley de Planck, con las mismas temperaturas y tomando 5 frecuencias para su interpolación (incluir $\nu=0$ y $\nu = \nu_{max}$, una frecuencia entre estas dos, y dos frecuencias posteriores a la frecuencia del máximo). Compare resultados con interpolación por \textit{splines} cúbicos y otra técnica de interpolación, y el resultado del numeral anterior.
\end{enumerate}

Información adicional:
\begin{itemize}
\item \href{http://en.wikipedia.org/wiki/Black-body_radiation#Equations}{Radiación de cuerpo negro}.
\item \href{http://en.wikipedia.org/wiki/Stefan%E2%80%93Boltzmann_law#Derivation_from_Planck.27s_law}{Ley de Stefan-Boltzmann}.
\item \href{http://en.wikipedia.org/wiki/Rayleigh%E2%80%93Jeans_law}{Ley de Rayleigh Jeans}.
\item \href{http://en.wikipedia.org/wiki/Wien%27s_displacement_law}{Ley de Wien}.
\end{itemize}

\end{document}